\documentclass[a4paper,12pt]{book}
\usepackage{graphicx}
\usepackage{float}
\usepackage[T1]{fontenc}
\usepackage{hyperref}
\usepackage{adjustbox}
\graphicspath{ {./images/} }
\pagenumbering{gobble}
\begin{document}

\author{Miłosz Wojciechowski}
\title{GoPro MAX video parser manual}
\date{\today}


\maketitle
\pagebreak
\pagenumbering{arabic}
\renewcommand{\labelenumii}{\arabic{enumi}.\arabic{enumii}}
\tableofcontents
\chapter{Introduction}
GoPro MAX video parser manual basic functionality is to extract equirectangular frames (every N frames or every N seconds) from a GoPro MAX spherical video and to do so the only things required are this program and an equirectangular video file.\\

\underline{Basic input is:} video file, directory to save frames, interval defining step between saved frames (in seconds or in frames).\\

\underline{Basic output:} folder with extracted frames.\\

Basic functionality may be expanded by using GoPro Telemetry extractor program which, if used, results in a total of:
\begin{itemize}
	\item Frames extraction
	\item Telemetry extraction
	\item Data visualization on a map
\end{itemize}

\underline{Extended input is:} basic input + GoPro Telemetry extractor program location and .LRV version of a video you are extracting frames from. LRV video has to be located in the same directory as the video you've chosen using "Choose video file" button. If you don't know what .LRV video is go to \hyperref[sec:video]{GoPro video types section}\\

\underline{Extended output is:} basic output + a csv file containing telemetry data (date, timestamp, latitude, longitude and altitude) and visualization on a map using map html file.

\pagebreak
List of requirements:
\begin{itemize}
	\item GoPro MAX camera
	\item Computer with at least a Windows 10 operating system
	\item Version control system Git
\end{itemize}
And for full functionality following additions are required:
\begin{itemize}
	\item Node js environment
	\item Telemetry extractor
\end{itemize}
To get full functionality of this program and receive map visualization follow instructions from \href{https://github.com/miloszwojciechowski/Open-vslam-project/tree/main/Manuals/Telemetry_extractor}{Telemetry Extractor manual} before proceeding further. If you've followed Telemetry Extractor manual, you can skip chapters 2 and 3 (Update GoPro MAX camera, Recording a video).
\chapter{Update GoPro MAX camera}
First of all you need to update your GoPro MAX before recording anything, following steps will show how to do this:
\begin{enumerate}
	\item Download and install the GoPro app on your mobile device (from \href{https://apps.apple.com/us/app/gopro-app/id561350520}{Apple App Store} on iOS or \href{https://play.google.com/store/apps/details?id=com.gopro.smarty&hl=en}{Google Play Store} on Android)
	\item Ensure that your camera is fully charged or with at least \verb|80%| remaining power
	\item Pair your camera with the GoPro app:
	\begin{itemize}
		\item Open GoPro app on your mobile device
		\item \begin{minipage}[t]{\linewidth}
			\raggedright
			\adjustbox{valign=t}{%
				\includegraphics[width=.8\linewidth]{gopro_update1}%
			}		
			\medskip	
		\end{minipage}
		Go to GoPro tab and click Connect a GoPro, choose MAX camera and follow the instructions
		\item If an update is available, the GoPro app will prompt you to update your camera
	\end{itemize}
\end{enumerate}

\chapter{Recording a video}
\section{How to record a video}
\begin{enumerate}
	\item Record a video with your GoPro camera using the following modes:
	\begin{itemize}
		\item Traditional video in HERO or 360 mode
		\item Time Lapse in HERO or 360 mode
	\end{itemize}
	How to use GoPro MAX camera is described in the GoPro MAX manual (link in the introduction). Make sure that you have turned on GPS function otherwise video won't have telemetry data:
	\begin{figure}[H]
		\centering
		\includegraphics{GoPro_manual_fragment}
		\caption{GoPro Max manual fragment.}
	\end{figure}
	If GPS is off, swipe down to access the Dashboard and Preferences. Click on the Preferences, find option Regional and there turn the GPS On.\\
\end{enumerate}

\section{GoPro video types}
\label{sec:video}
File structure of GoPro Max videos:\\	
GoPro Max creates different types of files during recording, in 360 mode we get:
\begin{itemize}
	\item .360 file (main video file)
	\item .THM file (thumbnail file)
	\item .LRV file (low-res video file)\\
\end{itemize}
In HERO mode (traditional video in 1080p or 1440p) we get:
\begin{itemize}
	\item .MP4 file (main video file)
	\item .THM file (thumbnail file)
	\item .LRV file (low-res video file)\\
\end{itemize}
These files always appear after recording a classical video or a Time Lapse.\\

\section{Export camera recordings}
\begin{enumerate}
	\item Turn on your GoPro camera.
	\item Open your GoPro MAX side panel and connect it to your computer using USB 2.0 to USB-c cable included in a camera set. Information as below should display on the screen:
	\begin{figure}[H]
		\centering
		\includegraphics{camera_connected}
		\caption{Successfully connected camera.}
	\end{figure} 
	
	\item Now find your connected GoPro camera and navigate through directories:\\
	
	$\textit{GoPro MAX > GoPro MTP Client Disk Volume > DCIM > 100GOPRO}$	\\
	
	Final path should look like this:
	
	\textit{GoPro MAX/GoPro MTP Client Disk Volume/DCIM/100GOPRO}	
	\begin{figure}[H]
		\centering
		\includegraphics{recording_location}
		\caption{Video files location.}
	\end{figure}
	\hfill
	\item 360 videos’ names and their LRV versions start with GS e.g. GS020007.360, GS020007.LRV.\\
	
	Regular videos’ and their LRV versions’ names start with GH for the former and with GL for the latter e.g. GH010008.MP4, GL010008.LRV. \\
	\item Copy videos of your choice and save them on your computer.
\end{enumerate}

\chapter{Install environment}
\section{Install FFmpeg}
\begin{enumerate}
	\item Go to \url{https://ffmpeg.org/download.html}
	\item \begin{minipage}[t]{\linewidth}
		\raggedright
		\adjustbox{valign=t}{%
			\includegraphics[width=.8\linewidth]{ffmpeg_install1}%
		}		
		\medskip	
	\end{minipage}
	To download Windows version click on the windows symbol and then choose Windows builds from gyan.dev
	\item \begin{minipage}[t]{\linewidth}
		\raggedright
		\adjustbox{valign=t}{%
			\includegraphics[width=.8\linewidth]{ffmpeg_install2}%
		}		
		\medskip	
	\end{minipage}
	On the new site choose latest git master branch build and download full version of ffmpeg
	\item Unpack the downloaded zip file in a directory of your choice
	\item Add FFmpeg to PATH:
	\begin{itemize}
		\item Press Windows+R and type "sysdm.cpl", window will pop up, go to "Advanced" tab and there press "Environment variables".
		\item \begin{minipage}[t]{\linewidth}
			\raggedright
			\adjustbox{valign=t}{%
				\includegraphics[width=.8\linewidth]{python_install6}%
			}		
			\medskip	
		\end{minipage}
		Window as above should pop up, click on the Path.
		\item \begin{minipage}[t]{\linewidth}
			\raggedright
			\adjustbox{valign=t}{%
				\includegraphics[width=.8\linewidth]{ffmpeg_install3}%
			}		
			\medskip	
		\end{minipage}
		Click New and paste path to \verb|\ffmpeg\bin| which contains ffmpeg.exe. Then confirm changes by pressing ok. After adding ffmpeg to PATH restart your computer.
		\item To check if installed correctly open Command Prompt (by writing cmd in the start menu) and write "ffmpeg"
	\end{itemize}
\end{enumerate}
\section{Install Python (at least version 3.9)}
\begin{enumerate}
	\item Download python  from \url{https://www.python.org/downloads/}. 
	\item \begin{minipage}[t]{\linewidth}
		\raggedright
		\adjustbox{valign=t}{%
			\includegraphics[width=.8\linewidth]{python_install1}%
		}		
		\medskip	
	\end{minipage}
	You can download the latest version by simply pressing the yellow button seen in the picture above. If you want an older version scroll down to the table under "Looking for a specific release?". But not all of them offer installers to download - last python 3.9 release with installer to download is 3.9.13. 
	\item \begin{minipage}[t]{\linewidth}
		\raggedright
		\adjustbox{valign=t}{%
			\includegraphics[width=.8\linewidth]{python_install2}%
		}		
		\medskip	
	\end{minipage}
	Choose your installer, either 32-bit or 64-bit depending on your system architecture.
	\item After downloading run the installer
	\item \begin{minipage}[t]{\linewidth}
		\raggedright
		\adjustbox{valign=t}{%
			\includegraphics[width=.8\linewidth]{python_install3}%
		}		
		\medskip	
	\end{minipage}
	Choose Customize Installation, check option "Add Python to PATH" and decide if you want to install Python for all users, I prefer not to.
	\item \begin{minipage}[t]{\linewidth}
		\raggedright
		\adjustbox{valign=t}{%
			\includegraphics[width=.8\linewidth]{python_install4}%
		}		
		\medskip	
	\end{minipage}
	Leave everything checked and again decide whether to install it for all users.
	\item \begin{minipage}[t]{\linewidth}
		\raggedright
		\adjustbox{valign=t}{%
			\includegraphics[width=.8\linewidth]{python_install5}%
		}		
		\medskip	
	\end{minipage}
	Choose options as in the picture above and if you wish change the install location and click install. In the ending screen you can choose to disable PATH length limit, but that requires admin permissions.
	\item To check if python was installed properly write "python" in cmd, this should display python version and enable python console. To quit write Ctrl+Z and press Enter. If it doesn't work or Microsoft Store window with Python to download opens try following steps:
	\begin{itemize}
		\item Press Windows+R and type "sysdm.cpl", window will pop up, go to "Advanced" tab and there press "Environment variables".
		\item \begin{minipage}[t]{\linewidth}
			\raggedright
			\adjustbox{valign=t}{%
				\includegraphics[width=.8\linewidth]{python_install6}%
			}		
			\medskip	
		\end{minipage}
		Window as above should pop up, click on the Path.
		\item \begin{minipage}[t]{\linewidth}
			\raggedright
			\adjustbox{valign=t}{%
				\includegraphics[width=.8\linewidth]{python_install7}%
			}		
			\medskip	
		\end{minipage}
		Make sure that all paths containing Python are above highlighted path to WindowsApps. If not, move them above WindowsApps path and restart your computer. Then repeat step 8 to check if now python works.
	\end{itemize}
\end{enumerate}
\pagebreak
\section{Install IDE (Integrated Development Environment)}
Below is provided an installation instruction for PyCharm, but you can install IDE of your choice.
\begin{enumerate}
	\item Go to \url{https://www.jetbrains.com/pycharm/download/?section=windows}
	\item \begin{minipage}[t]{\linewidth}
		\raggedright
		\adjustbox{valign=t}{%
			\includegraphics[width=.8\linewidth]{pycharm_install1}%
		}		
		\medskip	
	\end{minipage}
	Scroll down to download free Community Edition and click download button.
	\item \begin{minipage}[t]{\linewidth}
		\raggedright
		\adjustbox{valign=t}{%
			\includegraphics[width=.8\linewidth]{pycharm_install2}%
		}		
		\medskip	
	\end{minipage}
	Run the installer and choose install location.
	\item \begin{minipage}[t]{\linewidth}
		\raggedright
		\adjustbox{valign=t}{%
			\includegraphics[width=.8\linewidth]{pycharm_install3}%
		}		
		\medskip	
	\end{minipage}
	Choose options as above as they grant conveniency of usage, create desktop shortcut as you'd like.
	\item Then just click Install.
\end{enumerate}
\chapter{Prepare equirectangular video}
\begin{enumerate}
	\item Download and install Go Pro Player app \url{https://gopro.com/en/us/info/gopro-player}
	\item If you don't have HEVC video extension, use the installer located in \verb|HEVC_video_extension| provided within the main repository.
	\item Open .360 video in the GoPro Player
	 \item \begin{minipage}[t]{\linewidth}
	 	\raggedright
	 	\adjustbox{valign=t}{%
	 		\includegraphics[width=.8\linewidth]{player1}%
	 	}		
	 	\medskip	
	 \end{minipage}
	 Go to File
	 \item \begin{minipage}[t]{\linewidth}
	 	\raggedright
	 	\adjustbox{valign=t}{%
	 		\includegraphics[width=.8\linewidth]{player2}%
	 	}		
	 	\medskip	
	 \end{minipage}
	 Choose Export as > 5.6K
	 \item \begin{minipage}[t]{\linewidth}
	 	\raggedright
	 	\adjustbox{valign=t}{%
	 		\includegraphics[width=.8\linewidth]{player3}%
	 	}		
	 	\medskip	
	 \end{minipage}
	 Disable World Lock, make sure 5.6K and CineForm are chosen. Click Next, choose directory where equirectangular video will be created and wait for the Player to do the job.
\end{enumerate}
\chapter{Run the program}
\section{Create project}
\begin{enumerate}
	\item If you had followed telemetry-extraction manual, you should have a folder containing \verb|GoPro_MAX_video_parser| and GoPro-Telemetry-Extractor. If you don't have them, just download main repository: \url{https://github.com/miloszwojciechowski/Open-vslam-project} and extract it in a directory of your choice.
	\item \begin{minipage}[t]{\linewidth}
		\raggedright
		\adjustbox{valign=t}{%
			\includegraphics[width=.8\linewidth]{run1}%
		}		
		\medskip	
	\end{minipage}
	When in \verb|GoPro_MAX_video_parser| folder right-click and choose Open Folder as PyCharm Project.
	\item \begin{minipage}[t]{\linewidth}
		\raggedright
		\adjustbox{valign=t}{%
			\includegraphics[width=.8\linewidth]{run2}%
		}		
		\medskip	
	\end{minipage}
	A windows asking us how we would like to open the project should pop up, just click Trust Project.
	\item \begin{minipage}[t]{\linewidth}
		\raggedright
		\adjustbox{valign=t}{%
			\includegraphics[width=.8\linewidth]{run3}%
		}		
		\medskip	
	\end{minipage}
	Now we create virtual environment in the same folder, interpreter should be python that was installed earlier and dependencies the requirements.txt file. Click ok if everything is set.
	\linebreak
	\item \begin{minipage}[t]{\linewidth}
		\raggedright
		\adjustbox{valign=t}{%
			\includegraphics[width=.8\linewidth]{run4}%
		}		
		\medskip	
	\end{minipage}
	On the bottom of the screen you should see loading bar informing about building virtual environment and installing packages. When it finishes open main.py and run it.
		
	\item \begin{minipage}[t]{\linewidth}
		\raggedright
		\adjustbox{valign=t}{%
			\includegraphics[width=.8\linewidth]{window1}%
		}		
		\medskip	
	\end{minipage}
	After running main.py application window should open.
	There can be seen a few buttons and an input area:
	\begin{itemize}
		\item \textbf{Choose video file} - open file explorer and select video file. As opposed to telemetry extractor it is recommended to avoid .LRV file since extracted frames should be the highest possible quality.
		\item \textbf{Video type} - choose whether you parse normal video or time lapse (for the difference check \url{https://gopro.com/content/dam/help/max/manuals/MAX_UM_ENG_REVB.pdf}). If time lapse is chosen, pass the interval (in seconds) you chose on your GoPro camera (default is 0.5s), additionaly the saved frame step choice is locked and the only possible one is Frames.
		\item \textbf{Choose Frames save director}y - open file explorer and choose directory which extracted frames will be saved to. Make sure chosen folder is empty or create a new empty one.
		\item \textbf{Saved frame step} - first decide whether passed value should be number of seconds between extracted frames or number of frames between extracting extracted frames  e.g. when set to Seconds and number 30 is passed the program will extract one frame every 30 seconds but when set to Frames the program will extract one frame every 30 frames, so for video recorded in 30 FPS program will extract a frame every one second.
		\item \textbf{Pass the extractor.js localization} - to use only when Telemetry extractor was installed. Open file explorer and choose extractor.js localization (when using program with this function remember to put both .mov and .LRV files in the same folder), if not passed,  warning will pop up asking whether to run the application without extractor.js. If answered yes, only frames will be extracted.
		\item \textbf{Start} - run the program, before that make sure you filled all the needed data (video file, save directory and frame step). If running with extractor.js make sure that .LRV file is in the same directory as video file you provided via "Choose video file" button.
	\end{itemize}
	\item \begin{minipage}[t]{\linewidth}
		\raggedright
		\adjustbox{valign=t}{%
			\includegraphics[width=.8\linewidth]{window2}%
		}		
		\medskip	
	\end{minipage}
	After filling all the needed data press "Run" button. A console will open showing frames extraction process. If you want to stop it, press "Q". If extractor.js was provided csv file with telemetry data will appear in the folder with the provided video file and the map will open.\\
	If you run into \textbf{Couldn't extract telemetry data} error it might be caused by one of the below:
	\begin{itemize}
		\item No .LRV file in the folder
		\item Csv file telemetry is saved to is not closed
	\end{itemize}
\end{enumerate}
\end{document}